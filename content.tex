\section{First Section}     % Ab hier Inhalt

Für einen Hauptpunkt.

\section{Beipiel 2}

Hier ist der zweite Hauptpunkt.

\subsection{Beispiel 2.1}

Hier ist ein Beispiel für ein eingefügtes Bild.

\begin{figure}
    \centering
    \includegraphics{resources/pictures/logo.jpg}
    \caption{Das ist ein Logo}
    \label{fig:enter-label}
\end{figure}

\subsection{Beispiel 2.2}

Hier ist ein Beispiel für eine Tabelle.

\begin{table}
\caption{Hier eine Tabelle mit Beispiel-Werten}
\label{tab:Beispiel}
\begin{tabular}{l|l|l}
          & I in mA & U in V  \\ \hline
alle      & 1,41    & 25,074  \\ \hline
Q1        & 0,0123  & -20,574 \\
Q2        & 0,744   & 24,275  \\
Q3        & 0,255   & 23,364  \\
          &         &         \\
errechnet & 1,0113  & 27,065 
\end{tabular}
\end{table}

Die Tabelle kann natürlich auch noch schöner erstellt werden. Dafür empfiehlt es sich allerdings sich einzulesen.

\subsubsection{Beispiel 2.2.1}

Man kann auch noch tiefer.
Jetzt ein Beispiel zu Codes.

Zu Beginn mal ein Code-Beispiel via minted.

\inputminted{systemverilog}{systemVerilogDemo.sv}

\paragraph{Beispiel 2.2.1.1}

Auch Paragraphen sind möglich.

Hier jetzt noch ein Beispiel, wie man .tex-Files einbindet.
In dieser Vorlage ist dies auch mit .tex-File mit Präambel möglich.

% This LaTeX was auto-generated from MATLAB code.
% To make changes, update the MATLAB code and export to LaTeX again.

\documentclass{article}

\usepackage[utf8]{inputenc}
\usepackage[T1]{fontenc}
\usepackage{lmodern}
\usepackage{graphicx}
\usepackage{color}
\usepackage{hyperref}
\usepackage{amsmath}
\usepackage{amsfonts}
\usepackage{epstopdf}
\usepackage[table]{xcolor}
\usepackage{matlab}
\usepackage[paperheight=795pt,paperwidth=614pt,top=72pt,bottom=72pt,right=72pt,left=72pt,heightrounded]{geometry}

\sloppy
\epstopdfsetup{outdir=./}
\graphicspath{ {./MT02_media/} }

\begin{document}

\begin{matlabcode}
clear
clc
\end{matlabcode}


\begin{matlabcode}
A = 0.25^2*pi
\end{matlabcode}
\begin{matlaboutput}
A =    196.3495e-003
\end{matlaboutput}
\begin{matlabcode}
t=1.06
\end{matlabcode}
\begin{matlaboutput}
t =      1.0600e+000
\end{matlaboutput}
\begin{matlabcode}
t_s = t/sqrt(10)
\end{matlabcode}
\begin{matlaboutput}
t_s =    335.2014e-003
\end{matlaboutput}
\begin{matlabcode}
l=4e-2
\end{matlabcode}
\begin{matlaboutput}
l =     40.0000e-003
\end{matlaboutput}
\begin{matlabcode}
ro = 15.1
\end{matlabcode}
\begin{matlaboutput}
ro =     15.1000e+000
\end{matlaboutput}
\begin{matlabcode}
r_mine = ro*(l/A)
\end{matlabcode}
\begin{matlaboutput}
r_mine =      3.0761e+000
\end{matlaboutput}


\begin{matlabcode}
x = [1.867;1.94;1.74;1.76;1.94;1.739;1.67;1.679;1.834;1.893];
\end{matlabcode}


\begin{matlabcode}
r = x(:,1).*(A/l)
\end{matlabcode}
\begin{matlaboutput}
r = 10x1    
     9.1646e+000
     9.5230e+000
     8.5412e+000
     8.6394e+000
     9.5230e+000
     8.5363e+000
     8.1976e+000
     8.2418e+000
     9.0026e+000
     9.2922e+000

\end{matlaboutput}


\begin{matlabcode}
mittelwert = sum(x)/size(x,1)
\end{matlabcode}
\begin{matlaboutput}
mittelwert =      1.8062e+000
\end{matlaboutput}
\begin{matlabcode}
y = x
\end{matlabcode}
\begin{matlaboutput}
y = 10x1    
     1.8670e+000
     1.9400e+000
     1.7400e+000
     1.7600e+000
     1.9400e+000
     1.7390e+000
     1.6700e+000
     1.6790e+000
     1.8340e+000
     1.8930e+000

\end{matlaboutput}
\begin{matlabcode}
streuung = sqrt(sum((y(:,1)-mittelwert).*(y(:,1)-mittelwert))/(size(y,1)-1))
\end{matlabcode}
\begin{matlaboutput}
streuung =    101.9649e-003
\end{matlaboutput}
\begin{matlabcode}
vertrauen = streuung*t_s
\end{matlabcode}
\begin{matlaboutput}
vertrauen =     34.1788e-003
\end{matlaboutput}


\begin{matlabcode}

\end{matlabcode}

\end{document}


Für von Matlab exportierte .tex-Files ist auch das Highlighting konfiguriert.

\newpage                    % Anfang Verzeichnisse, nicht benötigte auskommentieren
\input{global/index/\lang}